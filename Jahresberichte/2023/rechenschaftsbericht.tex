% !TEX root = rechenschaftsbericht.tex
\newcommand{\jahr}{2023}
\documentclass[a4paper]{article}

\usepackage[T1]{fontenc}
\usepackage[utf8]{inputenc}
\usepackage{graphicx}
\usepackage{xcolor}
\usepackage{fancyhdr}

\graphicspath{{../../../materials/Vorlagen/}{images/}}
\lhead{Rechenschaftsbericht \jahr}
\rhead{\includegraphics[height=1em]{de-RSE-logo-text-colour}}
\pagestyle{fancy}

\usepackage{soul}
\usepackage{longtable}

\usepackage[breaklinks=true]{hyperref}
\def\UrlBreaks{\do\/\do-\do\ }

\usepackage{graphicx}

\renewcommand{\figurename}{Abbildung}



\begin{document}
\thispagestyle{empty}

\begin{centering}
\includegraphics[height=3em]{de-RSE-logo-text-colour}\\
\vspace{3em}
\textbf{
 \Large Rechenschaftsbericht des Vorstands\\*[.5em]
 \normalsize Geschäftsjahr: \jahr}\\*[3em]
\end{centering}

\section{Mitglieder und Mitgliedsbeiträge}

% TODO: Remove Beginn des Geschäftsjahres 60 = Ende 2022
Der Verein hatte zu Beginn des Geschäftsjahres 60 und zum Ende des Geschäftsjahres 69?? Mitglieder.

\section{Vorstand}

Dieser Rechenschaftsbericht wird vom Vorstand des Geschäftsjahres 2023 vorgelegt, welcher sich aus folgenden Personen zusammensetzt:

\begin{itemize}
  \setlength{\itemsep}{0pt plus 1pt}
  \item Jan Linxweiler (Vorsitzender)
  \item Frank Löffler (stellvertretender Vorsitzender)
  \item Jan Philipp Dietrich (Schriftführer)
  \item Bernadette Fritzsch (stellvertretende Schriftführerin)
  \item Stephan Janosch (Schatzmeister)
  \item Michael Meinel (stellvertretender Schatzmeister)
\end{itemize}

\section{Ereignisse im Zeitverlauf}

\begin{itemize}
 \item \textbf{1. Quartal}
   \begin{itemize}
     \item Mit einer gemeinsamen Absichtserklärung haben de-RSE e.V. - Gesellschaft für Forschungssoftware und die Gesellschaft für Informatik e.V. (GI) eine engere Zusammenarbeit vereinbart, um übereinstimmende Ziele zu erreichen und nach außen zu vertreten.
     \item Ende Februar kamen in Paderborn, passenderweise im Heinz Nixdorf MuseumsForum, 148 RSEs zur deRSE23-Konferenz zusammen.
     \item Der Verein hat sich mit einer \href{https://github.com/DE-RSE/Rueckmeldung_Konsultation_Forschungsdatengesetz/blob/master/Abgabe/R%C3%BCckmeldung_deRSE_eV.pdf}{Stellungnahme} an der \href{https://www.bmbf.de/SharedDocs/Downloads/de/2023/230306-forschungsdatengesetz-Einladungsschreiben.pdf?__blob=publicationFile&v=1}{öffentliche Konsultation zum Forschungsdatengesetz des BMBFs} beteiligt, um auf die Wichtigkeit der Einbeziehung von Forschungssoftware hinzuweisen.
     \item Die \href{https://github.com/DE-RSE/satzung/blob/40d49ffa9550cdbcbfb3b20284fc829c40cf41d0/de-RSE-e.V._Satzung.md}{Vereinssatzung in gegenderter Sprache} wird vom Vorstand einstimmig beschlossen. 

   \end{itemize}\clearpage
 \item \textbf{2. Quartal}
   \begin{itemize}
    \item Die Zusammenarbeit mit der GI wird aufgenommen und intensiviert.
    \item GI e.V. und de-RSE gründen \href{https://de-rse.org/blog/2023/07/18/RSE-Fachgruppe-de.html}{RSE Fachgruppe}.
    \item Es werden \href{https://fg-rse.gi.de/fachgruppe/arbeitskreise}{gemeinsame (themenspezifische) Arbeitsgruppen} mit der GI gebildet.
    \item Eine Restrukturierung der deRSE Infrastruktur wird beschlossen.
    \item Die Organisation der \dq un-deRSE23\dq{} ist zentrales Thema.

   \end{itemize}
 \item \textbf{3. Quartal}
   \begin{itemize}

    \item Der Verein ist bei der ersten NFDI-Konferenz "CoRDI" vertreten und organisiert \href{https://pad.gwdg.de/p/6uV8_9p8h#/}{RSE meet-up} (“Birds of a Feather” session). % https://pad.gwdg.de/Pj_OntLdRZyImG96wMhbzQ
    \item Der Verein listet das in Entwicklungen befindliche Positionspapier \href{https://de-rse.org/de/positions.html#work-in-progress}{\dq Establishing RSE departments in German research institutions\dq{}} auf seiner Webseite.
    \item Die \dq un-deRSE23\dq{} findet in Jena statt in deren Rahmen auch die Mitgliederversammlung statt findet und ein neuer Vorstand gewählt wird.
    \item Die Idee der Arbeitskreise aus der Zusammenarbeit mit der GI wird auf den Verein übertragen.
    
   \end{itemize}
 \item \textbf{4. Quartal}
   \begin{itemize}
    \item \href{https://fg-rse.gi.de/mitteilung/fachgruppentreffen-2023}{Ersten Treffen der RSE Fachgruppe} findet in Hannover statt.
    \item \href{https://fg-rse.gi.de/mitteilung/1-treffen-des-arbeitskreises-kategorien-von-forschungssoftware}{Erstes Treffen des Arbeitskreises Kategorien von Forschungssoftware} findet in Braunschweig statt.
    \item Es werden weitere \href{https://de-rse.org/de/working_groups.html}{Arbeitskreise auf Vereinsebene} eingerichtet.
    \item Der \href{https://de-rse.org/de/matrix.html}{Vereins-Chat} wird von RocketChat auf Matrix umgestellt.
    \item Es wird ein Sprecher*innen-Rat für die Arbeitskreise eingerichtet.
   \end{itemize}
\end{itemize}
\clearpage
\section{Weitere Ereignisse}

\begin{itemize}
 \item \textbf{Blog}
 \begin{itemize}
  \item \href{https://de-rse.org/blog/2023/02/03/Gesellschaft-fuer-Informatik-und-de-RSE-wollen-enger-zusammenarbeiten.html}{Gesellschaft für Informatik e.V. und de-RSE wollen enger zusammenarbeiten}
  \item \href{https://de-rse.org/blog/2023/04/05/R%C3%BCckmeldung-des-deRSE-zum-Forschungsdatengesetz.html}{Rückmeldung des deRSE zum Forschungsdatengesetz}
  \item \href{https://de-rse.org/blog/2023/04/25/deRSE23-de.html}{deRSE23 Konferenz}
  \item \href{https://de-rse.org/blog/2023/07/18/RSE-Fachgruppe-de.html}{GI e.V. und de-RSE gründen RSE Fachgruppe}
  \item \href{https://de-rse.org/blog/2023/10/05/rses-treffen-sich-auf-der-ersten-cordi-der-nfdi.html}{RSEs treffen sich auf der ersten CoRDI der NFDI}
  \item \href{https://de-rse.org/blog/2023/12/06/Performanceevaluation-fuer-Forschungssoftwareentwickler-de.html}{Performanceevaluation für Forschungssoftwareentwickler (RSEs)}
 \end{itemize}
 \item \textbf{Vorstandssitzungen}\\
  10 Vorstandssitzungen fanden an den folgenden Tagen statt: 12.1., 14.3., 25.4., 5.6., 20.6., 18.7., 29.8., 28.11., 14.12.
  Alle Sitzungen waren beschlussfähig.
  Protokolle sind öffentlich unter \href{https://github.com/DE-RSE/protokolle/}{https://github.com/DE-RSE/protokolle/} einsehbar.
 \item \textbf{OpenScience/deRSE-Community-Calls}\\
 Mitglieder des Vorstands beteiligen sich an allen 11 OpenScience/deRSE-Community-Calls im Jahr 2023.
\end{itemize}

% Mindestinhalt laut https://www.vereinswelt.de/rechenschaftsbericht
%Mitgliederentwicklung: Zu- und Abgang von Mitgliedern, Erläuterungen zu auffälligen Entwicklungen, Ausschlussverfahren
%Durchgeführte Vereinsveranstaltungen
%Teilnahme an Wettbewerben und Ergebnisse
%Beziehungen zum Dachverband und zu anderen Vereinen
%Stand laufender Projekte
%Struktur des Vereins
%Aktivitäten der Organe und Ausschüsse
%Sonstige Ereignisse, die für den Verein wichtig waren
%Finanzbericht
% Empfohlen
%Beziehungen zu Sponsoren und Spendern
%Aktivitäten zur Gewinnung weiterer Sponsoren und Spender
%Ausgang von für den Verein bedeutsamen Gerichtsverfahren
%Hauptamtliche Mitarbeiter, Veränderungen im Personalbestand
%Geplante Projekte und Aktivitäten


\end{document}
