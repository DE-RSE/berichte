% !TEX root = rechenschaftsbericht.tex
\newcommand{\jahr}{2024}
\newcommand{\todo}[1]{\textcolor{red}{ToDo: #1}}
\documentclass[a4paper]{article}

\usepackage[T1]{fontenc}
\usepackage[utf8]{inputenc}
\usepackage{graphicx}
\usepackage{xcolor}
\usepackage{fancyhdr}

\graphicspath{{../../../materials/Vorlagen/}{images/}}
\lhead{Rechenschaftsbericht \jahr}
\rhead{\includegraphics[height=1em]{de-RSE-logo-text-colour}}
\pagestyle{fancy}

\usepackage{soul}
\usepackage{longtable}

\usepackage[breaklinks=true]{hyperref}
\def\UrlBreaks{\do\/\do-\do\ }

\usepackage{graphicx}

\renewcommand{\figurename}{Abbildung}



\begin{document}
\thispagestyle{empty}

\begin{centering}
\includegraphics[height=3em]{de-RSE-logo-text-colour}\\
\vspace{3em}
\textbf{
 \Large Rechenschaftsbericht des Vorstands\\*[.5em]
 \normalsize Geschäftsjahr: \jahr}\\*[3em]
\end{centering}

\section{Mitglieder und Mitgliedsbeiträge}

Der Verein hatte zu Beginn des Geschäftsjahres 69 und zum Ende des Geschäftsjahres \todo{XXX} Mitglieder.

\section{Vorstand}

Dieser Rechenschaftsbericht wird vom Vorstand des Geschäftsjahres 2024 vorgelegt, welcher sich aus folgenden Personen zusammensetzt:

\begin{itemize}
  \setlength{\itemsep}{0pt plus 1pt}
  \item Jan Linxweiler (Vorsitzender)
  \item Frank Löffler (stellvertretender Vorsitzender)
  \item Jan Philipp Dietrich (Schriftführer)
  \item Bernadette Fritzsch (stellvertretende Schriftführerin)
  \item Stephan Janosch (Schatzmeister)
  \item Michael Meinel (stellvertretender Schatzmeister)
\end{itemize}

\section{Ereignisse im Zeitverlauf}

\begin{itemize}
 \item[] \textbf{1. Quartal}
   \begin{itemize}
     \item Die Umstellung des \href{https://de-rse.org/de/matrix.html}{Vereins-Chat} von RocketChat auf Matrix wird vollzogen und eine Dokumentation erstellt.
     \item Das Thema "Requirement Engineering in RSE" wird mit den Fachgruppen RSE und RE der GI in einem \href{https://fg-re.gi.de/veranstaltung/event-re-and-rse}{virtuellen Format} diskutiert.
     \item Die AG um das \href{https://doi.org/10.12688/f1000research.157778.2}{Positionspapier “Foundational Competencies and Responsibilities of a Research Software Engineer”} bendet die erste Schreibphase und startet die zweite Phase, in der das Papier von der Community diskutiert wird. 
     \item Der Arbeitskreis "RSE Software Development Guideline" nimmt seine Arbeit an einer "RSE Muster-Leitlinie" zur zur effizienten Entwicklung von Forschungssoftware auf.
     \item Anfang März kamen in Würzbug, \todo{XXX} RSEs zur deRSE24-Konferenz zusammen.
  
   \end{itemize}\clearpage
 \item[] \textbf{2. Quartal}
   \begin{itemize}
    \item Der Sprecher*innen-Rat für die Arbeitskreise ist eingerichtet und tagt des erste mal im Juni.
    \item Der Arbeitskreis "Teaching RSE" wird (gemeinsam mit der Gi) eingerichtet.
    \item Das \href{https://github.com/DE-RSE/2023_paper-RSE-groups}{Positionspapier “Establishing RSE departments in German research institutions”} ist von der AG zu 80\% abgeschlossen.
    \item Der Verein beteiligt sich an drei Projektanträgen im Rahmen des Förderprogramms \href{https://klaus-tschira-stiftung.de/foerderungen/naturwissenschaftliche-software/}{"Naturwissenschaftliche Software"} der Klaus Tschira Stiftung (KTS): \href{https://www.futursi.de/}{FutuRSI}, \href{https://the-teachingrse-project.github.io/RSE-Masters/}{RSE Master}, RSE Award (noch nicht gestartet).
     
    
   \end{itemize}
 \item[] \textbf{3. Quartal}
   \begin{itemize}

    \item Der gemeinsame Arbeitskreis von Gi und de-RSE um die \href{https://doi.org/10.18420/2025-gi_de-rse}{"RSE Muster-Leitlinie"} stellt seinen Entwurf vor und startet die Kommentierungsphase mit der Community.
    \item Die DFG stellt die Förderlinie "Forschungssoftwareinfrastrukturen" vor. Der Verein initiiert ein Treffen für den Austausch zwischen der Community und der DFG und richtet einen Matrix-Raum ein, um die Koordination von Anträgen zu unterstützen.
    \item Die Reviewphase mit der Community für das  \href{https://doi.org/10.12688/f1000research.157778.2}{Positionspapier “Foundational Competencies and Responsibilities of a Research Software Engineer”} ist bendet. Der Vorstand stimmt der Veröffentlichung als Positionspapier einstimmig zu.

   \end{itemize}
 \item[] \textbf{4. Quartal}
   \begin{itemize}
    \item Das zweite Treffen der RSE Fachgruppe der GI findet im Oktober in Köln statt.
    \item Der Review Prozess für die \href{https://doi.org/10.18420/2025-gi_de-rse}{"RSE Muster-Leitlinie"} wird abgeschlossen. Der Vorstand stimmt der Veröffentlichung einstimmig zu.
   \end{itemize}
\end{itemize}
\clearpage
\section{Weitere Ereignisse}

\begin{itemize}
 \item[] \textbf{Blog}
 \begin{itemize}
  \item \dots
 \end{itemize}
 \item[] \textbf{Vorstandssitzungen}\\
  14 Vorstandssitzungen fanden an den folgenden Tagen statt: 11.1., 15.2., 21.3., 18.4., 22.4., 16.5., 18.6., 18.7., 8.8., 15.8., 19.9., 18.10., 21.11., 19.12
  Alle Sitzungen mit Ausnahme der Sitzungen im August (8.8. und 15.8.) waren beschlussfähig.
  Protokolle sind öffentlich unter \href{https://github.com/DE-RSE/protokolle/}{https://github.com/DE-RSE/protokolle/} einsehbar.
 \item[] \textbf{OpenScience/deRSE-Community-Calls}\\
 Mitglieder des Vorstands beteiligen sich an \todo{allen 11} OpenScience/deRSE-Community-Calls im Jahr 2024.
\end{itemize}

% Mindestinhalt laut https://www.vereinswelt.de/rechenschaftsbericht
%Mitgliederentwicklung: Zu- und Abgang von Mitgliedern, Erläuterungen zu auffälligen Entwicklungen, Ausschlussverfahren
%Durchgeführte Vereinsveranstaltungen
%Teilnahme an Wettbewerben und Ergebnisse
%Beziehungen zum Dachverband und zu anderen Vereinen
%Stand laufender Projekte
%Struktur des Vereins
%Aktivitäten der Organe und Ausschüsse
%Sonstige Ereignisse, die für den Verein wichtig waren
%Finanzbericht
% Empfohlen
%Beziehungen zu Sponsoren und Spendern
%Aktivitäten zur Gewinnung weiterer Sponsoren und Spender
%Ausgang von für den Verein bedeutsamen Gerichtsverfahren
%Hauptamtliche Mitarbeiter, Veränderungen im Personalbestand
%Geplante Projekte und Aktivitäten


\end{document}
