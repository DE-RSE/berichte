\newcommand{\jahr}{2020}
\documentclass[a4paper]{article}

\usepackage[T1]{fontenc}
\usepackage[utf8]{inputenc}
\usepackage{graphicx}
\usepackage{xcolor}
\usepackage{fancyhdr}

\graphicspath{{../../../materials/Vorlagen/}{images/}}
\lhead{Rechenschaftsbericht \jahr}
\rhead{\includegraphics[height=1em]{de-RSE-logo-text-colour}}
\pagestyle{fancy}

\usepackage{soul}
\usepackage{longtable}

\usepackage[breaklinks=true]{hyperref}
\def\UrlBreaks{\do\/\do-\do\ }

\usepackage{graphicx}

\renewcommand{\figurename}{Abbildung}



\begin{document}
\thispagestyle{empty}

\begin{centering}
\includegraphics[height=3em]{de-RSE-logo-text-colour}\\
\vspace{3em}
\textbf{
 \Large Rechenschaftsbericht des Vorstands\\*[.5em]
 \normalsize Geschäftsjahr: \jahr}\\*[3em]
\end{centering}

\section{Mitglieder und Mitgliedsbeiträge}

Der Verein hatte zu Beginn des Geschäftsjahres 45 und zum Ende des Geschäftsjahres \textbf{TODO} Mitglieder. Von diesen \textbf{TODO} Mitgliedern bezahlten bisher \textbf{TODO} (Stand  \textbf{TODO}.8.2021) ihre Beiträge für das Jahr 2020.

\section{Vorstand}

Dieser Rechenschaftsbericht wird vom Vorstand des Geschäftsjahres 2020 vorgelegt, welcher sich aus folgenden Personen zusammensetzt:

\begin{itemize}
  \setlength{\itemsep}{0pt plus 1pt}
  \item Frank Löffler (Vorsitzender)
  \item Daniel Nüst (stellvertr. Vorsitzender)
  \item Bernadette Fritzsch (Schriftführerin)
  \item Stephan Druskat (stellvertretender Schriftführer)
  \item Stephan Janosch (Schatzmeister)
  \item Florian Thiery (stellvertretender Schatzmeister)
\end{itemize}

\section{Ereignisse im Zeitverlauf}

\begin{itemize}
 \item \textbf{8.1.}\\Der Vorstand trifft sich zu einem Ganztags-Präsenzvorstandstreffen in Berlin.
 \item \textbf{Januar}\\Die Planungen für die deRSE20-Konferenz laufen auf Hochtouren (u.a. Finanzierung, Finanzorganisation, Steuern, Räume).
 \item \textbf{21.1.}\\Nach ausführlicher Diskussion wurde folgender Beschluss gefasst: Die Konferenz deRSE20 wird vom Verein de-RSE e.V. direkt (und nicht unter dem Schirm der GI) ausgerichtet.
 Eine eigene Präsenz für deRSE-Chapter wird auf \href{https://de-rse.org/chapter/}{https://de-rse.org/chapter/} eingerichtet.
 \item \textbf{4.2.}\\Mietvertrag für deRSE20-Räume ist vorbereitet.
 \item \textbf{25.2.}\\Auf einer gemeinsamen Veranstaltung in Berlin haben sich Vertreterinnen und Vertreter von Fachkonsortien und Querschnittsinitiativen unter Beteiligung von de-RSE erneut über die Handlungsfelder der NFDI-übergreifenden Infrastrukturentwicklung ausgetauscht. Dabei haben Fachkonsortien und Querschnittsinitiativen vier modellhafte Vorschläge erarbeitet, um diese Handlungsfelder zu erweitern und im Rahmen der NFDI belastbar und nachhaltig umzusetzen. Diese „Leipzig-Berlin-Erklärung zu NFDI-Querschnittsthemen der Infrastrukturentwicklung“ wird im Juni veröffentlicht werden.
 \item \textbf{4.3.}\\Nach Sichtung mehrerer Angebote wird beschlossen (vor allem wegen der deRSE20-Konferenz) eine Haft- und Veranstaltungshaftpflichtversicherung abzuschließen. Neue Chapter wollen sich gründen und durch Präsenzveranstaltungen etablieren.
 \item \textbf{18.3.}\\Durch das nun volle Einsetzen der Corona-Krise werden viele Präsenzveranstaltungen abgesagt. Selbst die Durchführung der deRSE20-Konferenz im Sommer wird unklar. Durch den potentiellen Wegfall der Konferenz müssen Alternativpläne für die dort auch geplante Mitgliederversammlung gefunden werden.
 \item \textbf{1.4.}\\Es wurde entschieden, dass deRSE20 aufgrund der Corona-Krise nicht stattfinden wird. Alternativen werden aktiv diskutiert.
 \item \textbf{3.4.}\\Das Positionspapier "`An Environment for Sustainable Research Software in Germany and Beyond: Current State, Open Challenges, and Call for Action"' von Hartwig Anzt, Felix Bach, Stephan Druskat, Frank Löffler, Axel Loewe, Bernhard Y. Renard, Gunnar Seemann, Alexander Struck, Elke Achhammer, Piush Aggarwal, Franziska Appel, Michael Bader, Lutz Brusch, Christian Busse, Gerasimos Chourdakis, Peter Ebert, Bernd Flemisch, Sven Friedl, Bernadette Fritzsch, Maximilian D. Funk, Volker Gast, Florian Goth, Jean-Noël Grad, Sibylle Hermann, Florian Hohmann, Stephan Janosch, Dominik Kutra, Jan Linxweiler, Thilo Muth, Wolfgang Peters-Kottig, Fabian Rack, Fabian H.C. Raters, Stephan Rave, Guido Reina, Malte Reißig, Timo Ropinski, Joerg Schaarschmidt, Heidi Seibold, Jan P. Thiele, Benjamin Uekerman, Stefan Unger, Rudolf Weeber und Piotr Wojciech Dabrowski - vorliegend als \href{https://github.com/DE-RSE/positions/blob/8304df05448f22ae4293bb06ad513bc69a4ccc00/001/manuscript.pdf}{PDF} - wird einstimmig als offizielle Position von de-RSE e.V. angenommen.
 \item \textbf{22.4.}\\SORSE (Series of Online Research Software Events) wird als internationale Alternative zu ausfallenden RSE-Konferenzen besprochen.
 de-RSE e.V. beteiligt sich außerdem offiziell an der öffentlichen Konsultation zur Datenstrategie der Bundesregierung.
 \item \textbf{27.4.}\\Das Positionspapier "`An Environment for Sustainable Research Software in Germany and Beyond: Current State, Open Challenges, and Call for Action"' wird auf F1000 zum Review \href{https://f1000research.com/articles/9-295/v1}{veröffentlicht}.
 \item \textbf{13.5.}\\Die virtuelle Teilnahme von Mitgliedern an der Mitgliederversammlung wird vorbereitet (Konferenz- und Wahlsysteme werden getestet). de-RSE übernimmt die Ausfallsicherung der Kostenübernahme von AV-Kosten für SORSE (für den Fall, dass Sponsoren ausfallen sollten).
 \item \textbf{27.5.}\\Pläne für die alleinige Beteiligung von NFDI4RSE an der 2. Ausschreibungsrunde müssen wegen des Auschlusses von Querschnittsthemen geändert werden und Alternativen werden diskutiert. Möglichkeiten der Beteiligung an der NFDI werden mit dem Direktorium besprochen. Der Vorstand beschließt einstimmig, die offizielle Postadresse des Vereins von der bestehenden Privatadresse auf eine Dienstadresse (am DLR) zu ändern.
 \item \textbf{2.7.}\\Der NFDI4RSE-Abstrakt für die NFDI-Konferenz 2020 ist \href{https://www.dfg.de/download/pdf/foerderung/programme/nfdi/nfdi_konferenz_2020/nfdi4rse_abstract.pdf}{auf der DFG NFDI Webseite} verfügbar.
 \item \textbf{23.7.}\\Die Option, für NFDI4RSE mit der GI zusammen einen Antrag einzureichen wird aufgrund der fortgeschrittenen Zeit für 2020 verworfen.
 \item \textbf{August}\\Die Vorbereitungen für die Mitgliederversammlung 2020 laufen auf Hochtouren. Die (erste) Steuererklärung konnte fristgereicht eingereicht werden.
 \item \textbf{27.8.}\\Die Mitgliederversammlung 2020 wird weitestgehend virtuell abgehalten. Das Protokoll dazu hier \href{https://github.com/DE-RSE/protokolle/blob/master/Mitgliederversammlungen/MV-deRSE-2020-08-27-V1.md}{hier} zu finden. Der Vorstand wird für 2019 durch 24 anwesende Mitglieder entlastet. Martin Hammitzsch kandidiert nicht mehr für den Vorstand. Der Rest des Vorstandes wird im Amt bestätigt. Das Amt von Martin Hammitzsch übernimmt Florian Thiery.
 \item \textbf{15.,16 und 30.9.}\\Stephan Druskat, Frank Löffler und Stephan Janosch nehmen am RSE int leaders workshop teil. Ein entsprechender \href{https://researchsoftware.org/2020/10/09/2nd-international-rse-leaders-workshop.html}{Blogpost} wird veröffentlicht.
 \item \textbf{12.11.}\\Frank Löffler stellt die deutsche RSE-Gemeinschaft auf dem Supercomputing workshop Research Software Engineers in HPC Workshop (RSE-HPC-2020) vor.
 \item \textbf{Dezember}\\Die Möglichkeit einer RSE-Konferenz 2021 wird besprochen, der Vorstand ist aber nicht zuversichtlich. Der Beitritt des de-RSE e.V. im NFDI e.V. zusammen mit der GI wird vorbereitet.
\end{itemize}

\section{Weitere Ereignisse}

\begin{itemize}
 \item \textbf{Blog}
 \begin{itemize}
  \item \href{https://de-rse.org/blog/2020/01/31/derse20-goes-bazaar.html}{deRSE20 Goes Bazaar}
  \item \href{https://de-rse.org/blog/2020/09/22/deRSE-MV.html}{de-RSE e.V. Jahreshauptversammlung 2020}
 \end{itemize}
 \item \textbf{Vorstandssitzungen}\\
  21 Vorstandssitzungen fanden an den folgenden Tagen statt: 8.1., 21.1., 4.2., 18.3., 1.4., 3.4., 22.4., 13.5., 27.5., 16.6., 2.7., 23.7., 6.8., 11.8., 20.8., 3.9., 21.9., 13.10., 10.11. und dem 15.12.
  Bis auf die Sitzung vom 6.8. waren alle Sitzungen waren beschlussfähig.
  Protokolle sind öffentlich unter https://github.com/DE-RSE/protokolle/ einsehbar.
% \item \textbf{wann}\\
%  Vorstellungen von de-RSE auf Veranstaltungen
\end{itemize}

% Mindestinhalt laut https://www.vereinswelt.de/rechenschaftsbericht
%Mitgliederentwicklung: Zu- und Abgang von Mitgliedern, Erläuterungen zu auffälligen Entwicklungen, Ausschlussverfahren
%Durchgeführte Vereinsveranstaltungen
%Teilnahme an Wettbewerben und Ergebnisse
%Beziehungen zum Dachverband und zu anderen Vereinen
%Stand laufender Projekte
%Struktur des Vereins
%Aktivitäten der Organe und Ausschüsse
%Sonstige Ereignisse, die für den Verein wichtig waren
%Finanzbericht
% Empfohlen
%Beziehungen zu Sponsoren und Spendern
%Aktivitäten zur Gewinnung weiterer Sponsoren und Spender
%Ausgang von für den Verein bedeutsamen Gerichtsverfahren
%Hauptamtliche Mitarbeiter, Veränderungen im Personalbestand
%Geplante Projekte und Aktivitäten


\end{document}
