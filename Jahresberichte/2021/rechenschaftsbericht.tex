\newcommand{\jahr}{2021}
\documentclass[a4paper]{article}

\usepackage[T1]{fontenc}
\usepackage[utf8]{inputenc}
\usepackage{graphicx}
\usepackage{xcolor}
\usepackage{fancyhdr}

\graphicspath{{../../../materials/Vorlagen/}{images/}}
\lhead{Rechenschaftsbericht \jahr}
\rhead{\includegraphics[height=1em]{de-RSE-logo-text-colour}}
\pagestyle{fancy}

\usepackage{soul}
\usepackage{longtable}

\usepackage[breaklinks=true]{hyperref}
\def\UrlBreaks{\do\/\do-\do\ }

\usepackage{graphicx}

\renewcommand{\figurename}{Abbildung}



\begin{document}
\thispagestyle{empty}

\begin{centering}
\includegraphics[height=3em]{de-RSE-logo-text-colour}\\
\vspace{3em}
\textbf{
 \Large Rechenschaftsbericht des Vorstands\\*[.5em]
 \normalsize Geschäftsjahr: \jahr}\\*[3em]
\end{centering}

\section{Mitglieder und Mitgliedsbeiträge}

Der Verein hatte zu Beginn des Geschäftsjahres \textbf{TODO} und zum Ende des Geschäftsjahres \textbf{TODO} Mitglieder. Von diesen  \textbf{TODO} Mitgliedern bezahlten bisher  \textbf{TODO} (Stand \textbf{TODO: Datum}) ihre Beiträge für das Jahr 2021.

\section{Vorstand}

Dieser Rechenschaftsbericht wird vom Vorstand des Geschäftsjahres 2021 vorgelegt, welcher sich aus folgenden Personen zusammensetzt:

\begin{itemize}
  \setlength{\itemsep}{0pt plus 1pt}
  \item Frank Löffler (Vorsitzender)
  \item Daniel Nüst (stellvertr. Vorsitzender)
  \item Bernadette Fritzsch (Schriftführerin)
  \item Jan Philipp Dietrich (stellvertretender Schriftführer)
  \item Stephan Janosch (Schatzmeister)
  \item Florian Thiery (stellvertretender Schatzmeister)
\end{itemize}

\section{Ereignisse im Zeitverlauf}

\begin{itemize}
 \item \textbf{26.1.}\\Der Vorstand trifft sich virtuell zur ersten Vorstandssitzung des Jahres. Durch die Coronalage im Jahr 2021 wird auch keine der weiteren Vorstandssitzungen in einem anderen Format abgehalten werden können. Die internationaler RSE-Umfrage wird vorbereitet. Möglichkeiten für ein nationales Event 2021 werden diskutiert (deRSE21) und entsprechende Möglichkeiten ausgelotet. Aufgrund der anhaltenden Coronasituation wird ein größeres Ereignis 2021 jedoch verworfen und stattdessen der Tag der Forschungssoftware im Oktober anvisiert.
 \item \textbf{26.1.}\\Das Positionspapier "`An Environment for Sustainable Research Software in Germany and Beyond: Current State, Open Challenges, and Call for Action"' von Hartwig Anzt, Felix Bach, Stephan Druskat, Frank Löffler, Axel Loewe, Bernhard Y. Renard, Gunnar Seemann, Alexander Struck, Elke Achhammer, Piush Aggarwal, Franziska Appel, Michael Bader, Lutz Brusch, Christian Busse, Gerasimos Chourdakis, Peter Ebert, Bernd Flemisch, Sven Friedl, Bernadette Fritzsch, Maximilian D. Funk, Volker Gast, Florian Goth, Jean-Noël Grad, Sibylle Hermann, Florian Hohmann, Stephan Janosch, Dominik Kutra, Jan Linxweiler, Thilo Muth, Wolfgang Peters-Kottig, Fabian Rack, Fabian H.C. Raters, Stephan Rave, Guido Reina, Malte Reißig, Timo Ropinski, Joerg Schaarschmidt, Heidi Seibold, Jan P. Thiele, Benjamin Uekerman, Stefan Unger, Rudolf Weeber und Piotr Wojciech Dabrowski - vorliegend als \href{https://github.com/DE-RSE/positions/blob/8304df05448f22ae4293bb06ad513bc69a4ccc00/001/manuscript.pdf}{PDF} - wird als Peer Review Artikel veröffentlicht: \href{https://doi.org/10.12688/f1000research.23224.2}{https://doi.org/10.12688/f1000research.23224.2}
\item \textbf{22.2.}\\Frank Löffler, Christian Busse und Stephan Janosh halten auf der \href{https://indico.desy.de/event/28294/contributions/94417/}{RDA-DE 2021} einen \href{http://doi.org/10.5281/zenodo.4564161}{Vortrag} zum Thema ``Forschungssoftware in Deutschland'' und stellen damit u. a. den Verein vor.
 \item \textbf{26.2.}\\Bei der zweiten Vorstandssitzung wird u.a. an einem MoU mit der Gesellschaft für Informatik gearbeitet. Es wird weiterhin (weiter-)diskutiert, inwieweit de-RSE als Verein ein Dach für kleinere ``Funds'' für RSE-Gruppen sein könnte. Im weiteren Verlauf des Jahres wird dies jedoch vor allem mit Hinsicht auf die Gemeinnützigkeit verworfen.
  \item \textbf{22.3.}\\Frank Löffler spricht mit dem Direktor der NFDI, Prof. Dr. York Sure-Vetter, u.a. über eine mögliche Mitgliedschaft des deRSE e.V. im NFDI e.V.
  \item \textbf{24.3.}\\Der Vorstand trifft sich mit Lars Grunske (HUB) und diskutiert Möglichkeiten eines RSE-Tracks auf der SE22-Konferenz. Der entsprende Vorstandsbeschluss wird am 11.5. einstimmig angenommen.
  \item \textbf{24.3.}\\Das \href{https://sorse.github.io/programme/finale/}{SORSE Finale} lässt eine mehr als halbjährige Vortragsserie zum Ende kommen, bei dem sich auch Mitglieder von deRSE beteiligten.
  \item \textbf{30.3.}\\Der de-RSE e.V. stellt \href{https://www.softwareheritage.org}{Software Heritage} ein \href{https://www.softwareheritage.org/2021/03/30/society-for-research-software-in-germany-de-rse/}{Testimonial} aus.
  \item \textbf{April}\\Der de-RSE e.V. unterstützt offiziell \href{https://fair-software.eu/endorse}{fair-software.eu}.
  \item \textbf{20.4.}\\Das erstes Treffen des NFDI4Culture ExpertInnen-Forum ``Nachhaltige Softwareentwicklung in NFDI4Culture'' findet mit starker Beteiligung der de-RSE-Community und mit Einladung von Daniel Nüst statt. Aus der Gruppe heraus erfolgt später auch eine Kommentierung des Entwurfs zur Aktualisierung des Programms ``Informationsinfrastrukturen für Forschungsdaten'' der DFG mit deutlichen Hinweisen bzgl. Forschungssoftware.
  \item \textbf{21.4.}\\Der de-RSE e.V. stellt offiziell einen Mitgliedsantrag beim NFDI e.V.
  \item \textbf{Mai/Juni}\\Es wird eine Diskussion innerhalb der de-RSE-Community angestoßen über die Stellung des Vereins bzgl. publiccode.eu. Es wird schnell klar, dass es eine diverse Meinungslage gibt.
  \item \textbf{25.6.}\\Ein erstes de-RSE-Gruppentreffen findet (virtuell) statt. Die 23 Teilnehmenden diskutieren vorranging über die Anliegen/Nöte/Bedarfe von RSE-Gruppen diskutiert.
  \item \textbf{Juli/August}\\Die Mitgliederhauptversammlung 2021 wird vorbereitet.
  \item \textbf{5.7.}\\Der Vorstand beschließt einstimmig, publiccode.eu zu unterstützen, die Position aber in einem Blogpost mit Details zu hinterlegen, um die Nuancen in der Community abzubildeen.
\item \textbf{September}\\Stephan Janosch hat auf dem 38. DV-Tagen zum Thema Forschungssoftware gesprochen: \href{https://doi.org/10.6084/m9.figshare.16706374}{Forschungssoftware-(EntwicklerInnen) in MPG}.
\item \textbf{2.9}\\Der de-RSE e.V. beschließt, für Teilnehmende der virtuellen \href{https://septembrse.society-rse.org/}{SeptembRSE}-Konferenz die Tagungsgebühren zu übernehmen. Von diesem Angebot wird im Folgenden kein Gebrauch gemacht.
  \item \textbf{6.9.}\\Jan Dietrich stellt de-RSE auf der virtuellen \href{https://septembrse.society-rse.org/}{SeptembRSE}-Konferenz vor.
 \item \textbf{15.9.}\\Der de-RSE e.V. führt die Jahreshauptversammlung 2021 durch. Hierbei wird Stephan Druskat aus dem Vorstand verabschiedet und Jan Dietrich in den Vorstand aufgenommen.
 \item \textbf{Oktober}\\Zwei de-RSE-Mitglieder tragen auf der WOSSS21 vor: Martin Hammitzsch: ``\href{https://wosss.org/wosss21/S4-MartinHammitzsch}{The fundamental part of software: the human infrastructure}'' und Alexander Struck: ``\href{https://wosss.org/wosss21/S4-AlexanderStruck}{(Inter)National Community Efforts by the German Association of Research Software Engineers (de-RSE)}''.
 \item \textbf{14.10.}\\Der Tag der Forschungssoftware 2021 wird durch ein virtuelles Treffen in Deutschland begangen. Themen kommen aus dem gesamten Lebenszyklus von Forschungssoftware, beinhalten die Bedeutung von Forschungssoftware in der NFDI, sowie RSE-Rollen und -Karrieren, RSE-Gruppen und RS in der MPI.
 \item \textbf{November}\\Die internationale RSE-Umfrage 2021 startet, u.a. mit Zuarbeit von deutscher Seite von Stephan Janosch.
 \item \textbf{23.11.}\\Frank Löffler stellt das Thema Forschungssoftware und de-RSE als Verein auf einem Treffen der ``common infractructures''-Sektion der NFDI vor.
 \item \textbf{November/Dezember}\\Der de-RSE e.V. beschließt, sich finanziell mit 2000\ € und personell (Gutachter/innen) am campus-source-Award 2021 zu beteiligen. Die Sichtung der Einreichungen erfolgt noch 2021, die Benennung der Gewinner jedoch erst 2022.
 \item \textbf{zweite Jahreshälfte}\\Der Forschungssoftwaretrack auf der SE22 wird vorbereitet, mit Einreichungen für 5 BoF/Splinter-Meetings, einem Lightning Talk, 3 Postern, 15 Vorträgen und 4 Workshops.
\end{itemize}

\section{Weitere Ereignisse}

\begin{itemize}
 \item \textbf{Blog}
 \begin{itemize}
  \item \href{https://de-rse.org/blog/2021/03/18/board-meeting-1.html}{Vorstandssitzungen im 1. Quartal 2021}
  \item \href{https://de-rse.org/blog/2021/04/26/aenderung-der-postadresse.html}{Änderung der Postadresse}
  \item \href{https://de-rse.org/blog/2021/05/20/nfdi-membership.html}{de-RSE e.V. - Gesellschaft für Forschungssoftware wird Mitglied im NFDI e.V.}
  \item \href{https://de-rse.org/blog/2021/06/01/rse-gruppentreffen.html}{Beginn von RSE-Gruppentreffen in Deutschland}
  \item \href{https://de-rse.org/blog/2021/06/10/save-the-date-MV-2021.html}{Save the date, werde Vorstand, leite die Wahl bei der Mitgliederversammlung 2021}
  \item \href{https://de-rse.org/blog/2021/06/19/rse-gruppentreffen.html}{Ankündigung: 1. Treffen von RSE-Gruppen in Deutschland}
  \item \href{https://de-rse.org/blog/2021/08/10/rse-gruppentreffen.html}{1. Treffen von RSE-Gruppen in Deutschland}
  \item \href{https://de-rse.org/blog/2021/09/02/septembRSE-Unterst\%C3\%BCtzung.html}{Der de-RSE e.V. unterstützt eure Teilnahme an der SeptembRSE-Konferenz im UK}
  \item \href{https://de-rse.org/blog/2021/10/04/SE2022-RSE-Track-Call-for-contributions-de.html}{rSE22 Call for Contributions}
 \end{itemize}
 \item \textbf{Vorstandssitzungen}\\
  15 Vorstandssitzungen fanden an den folgenden Tagen statt: 26.1., 26.2., 15.3., 13.4., 11.5., 9.6., 5.7., 27.7., 31.8., 2.9., 13.9., 23.9., 12.10., 2.11. und 6.12.
  Bis auf die Sitzungien vom 26.2. und 6.12. waren alle Sitzungen beschlussfähig.
  Protokolle sind öffentlich unter \href{https://github.com/DE-RSE/protokolle/}{https://github.com/DE-RSE/protokolle/} einsehbar.
 \item \textbf{OpenScience/deRSE-Community-Calls}\\
 Mitglieder des Vereins beteiligen sich an den 12 OpenScience/deRSE-Community-Calls der Tage: 13.1., 3.2., 3.3., 7.4., 5.5., 2.6., 7.7., 4.8., 1.9., 6.10., 3.11. und 1.12.
% \item \textbf{wann}\\
%  Vorstellungen von de-RSE auf Veranstaltungen
\end{itemize}

% Mindestinhalt laut https://www.vereinswelt.de/rechenschaftsbericht
%Mitgliederentwicklung: Zu- und Abgang von Mitgliedern, Erläuterungen zu auffälligen Entwicklungen, Ausschlussverfahren
%Durchgeführte Vereinsveranstaltungen
%Teilnahme an Wettbewerben und Ergebnisse
%Beziehungen zum Dachverband und zu anderen Vereinen
%Stand laufender Projekte
%Struktur des Vereins
%Aktivitäten der Organe und Ausschüsse
%Sonstige Ereignisse, die für den Verein wichtig waren
%Finanzbericht
% Empfohlen
%Beziehungen zu Sponsoren und Spendern
%Aktivitäten zur Gewinnung weiterer Sponsoren und Spender
%Ausgang von für den Verein bedeutsamen Gerichtsverfahren
%Hauptamtliche Mitarbeiter, Veränderungen im Personalbestand
%Geplante Projekte und Aktivitäten


\end{document}
