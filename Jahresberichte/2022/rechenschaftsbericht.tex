\newcommand{\jahr}{2022}
\documentclass[a4paper]{article}

\usepackage[T1]{fontenc}
\usepackage[utf8]{inputenc}
\usepackage{graphicx}
\usepackage{xcolor}
\usepackage{fancyhdr}

\graphicspath{{../../../materials/Vorlagen/}{images/}}
\lhead{Rechenschaftsbericht \jahr}
\rhead{\includegraphics[height=1em]{de-RSE-logo-text-colour}}
\pagestyle{fancy}

\usepackage{soul}
\usepackage{longtable}

\usepackage[breaklinks=true]{hyperref}
\def\UrlBreaks{\do\/\do-\do\ }

\usepackage{graphicx}

\renewcommand{\figurename}{Abbildung}



\begin{document}
\thispagestyle{empty}

\begin{centering}
\includegraphics[height=3em]{de-RSE-logo-text-colour}\\
\vspace{3em}
\textbf{
 \Large Rechenschaftsbericht des Vorstands\\*[.5em]
 \normalsize Geschäftsjahr: \jahr}\\*[3em]
\end{centering}

\section{Mitglieder und Mitgliedsbeiträge}

Der Verein hatte zu Beginn des Geschäftsjahres 56 und zum Ende des Geschäftsjahres XX Mitglieder. Von diesen XX Mitgliedern bezahlten bisher XX (Stand XX.XX.202X) ihre Beiträge für das Jahr 2022.

\section{Vorstand}

Dieser Rechenschaftsbericht wird vom Vorstand des Geschäftsjahres 2022 vorgelegt, welcher sich aus folgenden Personen zusammensetzt:

\begin{itemize}
  \setlength{\itemsep}{0pt plus 1pt}
  \item Frank Löffler (Vorsitzender)
  \item Daniel Nüst (stellvertr. Vorsitzender)
  \item Bernadette Fritzsch (Schriftführerin)
  \item Jan Philipp Dietrich (stellvertretender Schriftführer)
  \item Stephan Janosch (Schatzmeister)
  \item Florian Thiery (stellvertretender Schatzmeister)
\end{itemize}

\section{Ereignisse im Zeitverlauf}

\begin{itemize}
 \item \textbf{1. Quartal}
   \begin{itemize}
     \item de-RSE unterstützt den \href{https://ev.campussource.de/publikationen/csa2022/}{CampusSource Award 2022} mit 2000 €.
     \item Die RSE Arbeitsgruppe innerhalb der NFDI Sektion "`common infrastructures"' formiert sich.
     \item Die internationale RSE-Umfrage wird abgeschlossen.
   \end{itemize}
 \item \textbf{2. Quartal}
   \begin{itemize}
    \item de-RSE e.V. unterstützt offiziell \href{https://publiccode.eu}{publiccode.eu} (public money, public code).
    \item deRSE23 wird neben der SE23-Konferenz stattfinden.
    \item un-deRSE23 wird als inhaltlich anders gestaltetes Format in Jena stattfinden.
   \end{itemize}
 \item \textbf{3. Quartal}
   \begin{itemize}
% vielleicht sollten wir auch Vorträge aufführen, wo wir auf de-RSE hinweisen. hier als Vorschlag zumindest meine beiden:
    \item 18.10.2022 "Research Software - ideal vs real", Impulsvortrag auf Retreat am FZJ (B. Fritzsch) 
    \item 15.09.2022 "Reseach Software and the people behind it", HiRSE Seminar (B. Fritzsch)
   \end{itemize}
 \item \textbf{4. Quartal}
   \begin{itemize}
    \item .
   \end{itemize}
 \item \textbf{TODO}
\end{itemize}

\section{Weitere Ereignisse}

\begin{itemize}
 \item \textbf{Blog}
 \begin{itemize}
  \item \href{https://de-rse.org/blog/2023/02/03/Gesellschaft-fuer-Informatik-und-de-RSE-wollen-enger-zusammenarbeiten.html}{MoU mit Gesellschaft für Informatik e.V.}
  \item \href{https://de-rse.org/blog/2023/04/05/R\%C3\%BCckmeldung-des-deRSE-zum-Forschungsdatengesetz.html}{Rückmeldung des deRSE zum Forschungsdatengesetz}
  \item \href{https://de-rse.org/blog/2023/07/18/RSE-Fachgruppe-de.html}{GI e.V. und de-RSE gründen RSE Fachgruppe}
 \end{itemize}
 \item \textbf{Vorstandssitzungen}\\
  15 Vorstandssitzungen fanden an den folgenden Tagen statt: 18.1., 8.2. 8.3., 5.4., 11.5., 2.6., 23.6., 28.9., 3.11., 19.12.
  Bis auf die Sitzung am 11.5. waren alle Sitzungen beschlussfähig.
  Protokolle sind öffentlich unter \href{https://github.com/DE-RSE/protokolle/}{https://github.com/DE-RSE/protokolle/} einsehbar.
 \item \textbf{OpenScience/deRSE-Community-Calls}\\
 Mitglieder des Vorstands beteiligen sich an allen 11 OpenScience/deRSE-Community-Calls im Jahr 2022.
\end{itemize}

% Mindestinhalt laut https://www.vereinswelt.de/rechenschaftsbericht
%Mitgliederentwicklung: Zu- und Abgang von Mitgliedern, Erläuterungen zu auffälligen Entwicklungen, Ausschlussverfahren
%Durchgeführte Vereinsveranstaltungen
%Teilnahme an Wettbewerben und Ergebnisse
%Beziehungen zum Dachverband und zu anderen Vereinen
%Stand laufender Projekte
%Struktur des Vereins
%Aktivitäten der Organe und Ausschüsse
%Sonstige Ereignisse, die für den Verein wichtig waren
%Finanzbericht
% Empfohlen
%Beziehungen zu Sponsoren und Spendern
%Aktivitäten zur Gewinnung weiterer Sponsoren und Spender
%Ausgang von für den Verein bedeutsamen Gerichtsverfahren
%Hauptamtliche Mitarbeiter, Veränderungen im Personalbestand
%Geplante Projekte und Aktivitäten


\end{document}
